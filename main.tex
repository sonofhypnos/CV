\documentclass[letterpaper, 9pt]{article}

\usepackage[margin=0.40in]{geometry}
\usepackage{xcolor}
\usepackage[colorlinks = true, urlcolor = yellow]{hyperref}

% Define colors for dark mode
\definecolor{background}{RGB}{35,35,35}
\definecolor{text}{RGB}{220,220,220}
\definecolor{accent}{RGB}{100,149,237} % Light blue color

% Set page color and text color
\pagecolor{background}
\color{text}

\usepackage{titlesec}
\titleformat{\section}{\large\bfseries\color{accent}}{\thesection}{0.5em}{}[\titlerule]
\titlespacing*{\section}{0pt}{0.5em}{0.3em}

\usepackage{enumitem}
\setlist{nolistsep, leftmargin=*}

\input{flag.tex}


% Compact header style
\newcommand{\header}[4]{
    \begin{center}
        \textbf{\Huge{#1}}\\
        \small{
            \textcolor{accent}{#2} \hspace{1em}
            \href{#3}{GitHub}\hspace{1em}
            \href{#4}{tassilo.neubauer.com}
        }
    \end{center}
}

\begin{document}


\header{Tassilo Neubauer}{tassilo.neubauer@gmail.com}{https://github.com/sonofhypnos}{https://www.tassiloneubauer.com/}

\section*{Education}
\textbf{Karlsruhe Institute of Technology} \hfill Karlsruhe, Germany \\
B.S. Computer Science, Grade Average 3.1 on a scale from 5 (lowest) to 1 (highest) \hfill October 2020--December 2023
\begin{itemize}
    \item \textit{Thesis:} Machine Learning-based Root Cause Analysis for Intelligent Production.
\end{itemize}
% \textbf{Freie Waldorfschule Wahlwies} \hfill Stockach, Germany \\
% Abitur 1.4 (scale: 1.0 best - 4.0 failed) \hfill 2020
\textbf{Freie Waldorfschule Wahlwies} \hfill Abitur 1.4 (scale: 1.0 best - 4.0 failed) (2020)
% \textbf{Freie Waldorfschule Wahlwies} \hfill Stockach, Germany \\
% High school diploma,  1.4 on a scale from 5 (lowest) up to 1 (highest) grade \hfill September 2007--July 2020


\section*{Experience}
\begin{itemize}
    \item \textbf{Research Intern, Bootstrap Bio} \hfill Online, October 2024--Present
    \begin{itemize}
       % \it Conducting research on the epigenetic state in pre-implantation embryos
      \item Evaluated feasibility of methods for assessing embryo health post in-vitro culturing through Python-based analysis of published epigenetic datasets
      \item Identified and clearly communicated critical limitations in existing research data that impacted project direction
      \item Self-taught molecular biology concepts to effectively analyze scientific literature and evaluate technical approaches
      \item Worked independently in a remote setting, conducting literature review and data analysis with minimal supervision

    \end{itemize}
    \item \textbf{Freelance Software Developer, BoostX} \hfill Online, June 2024--August 2024
    \begin{itemize}
       \item Worked as a full-stack developer for BoostX, an early-stage startup. (Typescript, React, Node.js, MongoDB, SQLite)
       \item Contributed to their MVP that uses LLM technology to create personalized AI chatbots for content creators.
    \end{itemize}
    \item \textbf{Freelance Software Developer, Exosome Consulting} \hfill Online, January 2024--June 2024
    \begin{itemize}
       \item Worked on data pipeline development with Exosome Consulting. (Python, SQLite)
    \end{itemize}
    \item \textbf{Research Assistant, Forschungszentrum Informatik} \hfill Karlsruhe, Germany, March 2022--December 2022
    \begin{itemize}
        \item Generated visualization recommendations using evolutionary algorithms based on artificial feedback. (Python)
    \end{itemize}
    \item \textbf{Research Assistant, KIT Political Economy Chair} \hfill Karlsruhe, Germany, September 2021--December 2021
    \begin{itemize}
        \item Optimized online experiment deployment previously handled by another chair, reducing time to recruit participants from weeks to hours (using Prolific)
    \end{itemize}
\end{itemize}

	\section*{Skills}
	\noindent\small\textit{Proficiency rated on a scale of 1 (basic) to 3 (high).}
	\normalsize
	\begin{itemize}
		\item \textbf{Programming Languages:} Python (3), Typescript (2), SQL (2), Java (2), C++ (2), Bash (2), Nix (1), Haskell (1)
		\item \textbf{Software and Libraries:} Emacs (3), React (2), Numpy (2), Pandas (2), Pytorch (2), Latex (1), Make (1)
		\item \textbf{Operating Systems:} Linux (3), Windows (1)
		\item \textbf{Languages:} German (native), English (3), Japanese (1), French (1)
	\end{itemize}

\section*{Projects}
  \begin{itemize}
    \item \textbf{PIBBSS Research Hackathon: Belief State Steering in Transformer Models} \hfill June 1--3, 2024
    \begin{itemize}
        \item Participated in a 3-day research hackathon with two collaborators
        \item Compared effectiveness of steering across  different layers of the Transformer architecture
        \item Produced \href{https://github.com/sonofhypnos/CV/blob/main/steering-models-belief-states.pdf}{a brief technical report} on methods and findings
    \end{itemize}
    \item \textbf{\href{https://github.com/sonofhypnos/fermi}{GPT-4 Fermi Estimation Evaluation}} \hfill March 4--5, 2024
    \begin{itemize}
      \item Developed a Python script evaluating LLM performance on Fermi estimation tasks
      \item Achieved a mean fp-score of 0.57 on 100 samples, significantly outperforming previous research
      \item Revealed GPT-4's superior accuracy compared to existing benchmarks and dataset ground truths
    \end{itemize}
    \item \textbf{\href{https://github.com/sonofhypnos/fatebook.el}{fatebook.el}: Emacs Plugin for Fatebook} \hfill September 2023
    \begin{itemize}
        \item Developed an Emacs plugin to create and upload predictions to Fatebook for personal use
    \end{itemize}
    % \item \textbf{ArmarX Distributed Robot Framework Plugin} \hfill April 2022--September 2022
    % \begin{itemize}
    %     \item Collaborated with four students to develop a C++ GUI plugin for ArmarX
    %     \item Designed and implemented a foldable timeline visualizing robot's hierarchical memory
    % \end{itemize}
    \item \textbf{\href{https://sonofhypnos.github.io/blog/prediction/python/2021/01/30/sp500.html}{S\&P 500 Prediction Model}} \hfill 2020
    \begin{itemize}
        \item Built a Python model using NumPy and Pandas to predict S\&P 500 movements
        \item Successfully used the model to win 50£ on the \href{https://www.almanisprivate.com/}{Almanis} prediction market
    \end{itemize}
  \end{itemize}

\section*{Courses}
\begin{itemize}
    \item \textbf{Math:} Linear Algebra I \& II, Real Analysis I \& II, Probability Theory, Numerical Math for Computer Scientists
    \item \textbf{Computer Science:} Programming (Java), Software Engineering (Java), Programming Paradigms, Databases, Cognitive Systems, Theory of Computer Science I \& II, Algorithms I \& II, Advanced Algorithmic Programming, Operating Systems, Computer Networks among others
    \item \textbf{Physics:} Experimental Physics I \& II, Modern Physics for Computer Scientists
\end{itemize}

% TODO: include seminar and presentation on deep learning?

\section*{Extracurriculars}
  \begin{itemize}
      \item {\textbf{Co-organizer of local Effective Altruism group Karlsruhe}} \hfill April 2021--November 2023 \\
      {Booked rooms, went through the paperwork to become an official university group.}
      % \item {\textbf{Member in \href{https://web.archive.org/web/20220123065219/http://aisafety.com/}{AI-Safety-Reading-Group}}} \hfill October 2019--June 2020 \\
      % {Learned about risks from advanced artificial intelligence. Read Human Compatible, Superintelligence and Rationality from AI to Zombies.}
      \item {\textbf{Writing on \href{https://www.tassiloneubauer.com/}{my blog}}} \hfill 2021--Current
      \item \textbf{Forecasting on \href{https://www.metaculus.com/accounts/profile/106992/}{Metaculus}} \hfill June 2019--March 2020
      \begin{itemize}
              \item 11th most accurate predictor in 2019
              \item Made $\sim$2000 predictions on $\sim$600 questions on Metaculus
      \end{itemize}
  \end{itemize}

\end{document}
