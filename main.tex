\input{flag.tex}

\documentclass[letterpaper,9pt]{article}

\usepackage[style=ieee,url=false,doi=false,maxbibnames=99,sorting=ydnt,dashed=false]{biblatex}
\bibliography{papers}

% \usepackage{hyperref}

\def\theme{MidnightBlue}
\usepackage{simplecv}
\hypersetup{
	colorlinks=true,
	linkcolor=blue,
	filecolor=magenta,
	urlcolor=cyan,
	pdftitle={CV},
	pdfpagemode=FullScreen,
}



\boldname{Tassilo}{Neubauer}{N.}

\begin{document}
\headinginline{Tassilo Neubauer}{
	% Website: \website{www.tassiloneubauer.com} \\
	Email: \email{tassilo.neubauer@gmail.com} \\
	GitHub: \github{sonofhypnos}
}

% \headingphoto{Name Surname}{
%     Website: \website{example.com} \\
%     Email: \email{example@example.edu} \\
%     LinkedIn: \linkedin{name-surname} \\
%     GitHub: \github{example}
% }{photo.jpg}
\section{Education}
\outerlist{
	\entrybig{\textbf{Karlsruhe Institute of Technology}}{Karlsruhe, Germany}
	{B.S. Computer Science, Grade Average 3.1 on a scale from 5 (lowest) to 1 (highest).}{October 2020--December 2023}
	\begin{itemize}
		\item \textit{Thesis:} Machine Learning-based Root Cause Analysis for Intelligent Production.
	\end{itemize}

	% \entrybig{\textbf{Freie Waldorfschule Wahlwies}}{Stockach, Germany}
	% {High school diploma,  1.4 on a scale from 5 (lowest) to 1 (highest) grade} {2007--2020}
}
\section{Courses}
\denseouterlist{
	\entry{\textbf{Math:} Linear Algebra I \& II, Real Analysis I \& II, Probability Theory, Numerical Math for Computer Scientists}
	\entry{\textbf{Computer Science:} Programming (Java), Software Engineering (Java), Programming Paradigms, Databases, Cognitive Systems, Theory of Computer Science I \& II, Algorithms I \& II, Advanced Algorithmic Programming, Operating Systems, Computer Networks among others}
	\entry{\textbf{Physics:} Experimental Physics I \& II, Modern Physics for Computer Scientists}
}

\ifdefined\SkillVersion
	\section{Skills}
	\noindent\small\textit{Proficiency rated on a scale of 1 (basic) to 3 (high).}
	\normalsize
	\begin{itemize}
		\item \textbf{Programming Languages:} Python (3), Java (2), C++ (2), Bash (2), C (1), Nix (1), SQL (1), Haskell (1)
		\item \textbf{Software and Libraries:} Emacs (3), Numpy (2), Pandas (2), Pytorch (2), Latex (1), Make (1)
		\item \textbf{Operating Systems:} Linux (3), Windows (1)
		\item \textbf{Languages:} German (3 - native), English (3), Japanese (1), French (1)
	\end{itemize}
\fi

\section{Experience}
\begin{itemize}
	\item \textbf{Freelance Software Developer} \hfill Online, January 2024--Current \\
        Working on a data scraping job with Exosome Consulting.
	\item \textbf{Stanford Existential Risk Initiative} \hfill Online, November 2022--December 2022 \\
		Participated in John Wentworth's agent foundations training program. Conducted numerous exercises in conceptual research.
	\item \textbf{Forschungszentrum Informatik, Karlsruhe, Germany} \hfill March 2022--December 2022 \\
	      Research assistant focusing on generating visualization recommendations using an
	      evolutionary algorithm combined with random forests and simple feedforward neural nets,
	      based on artificial human feedback.
	\item \textbf{Karlsruhe Institute of Technology, Karlsruhe, Germany} \hfill April 2022--September 2022 \\
	      Collaborated with a team of four students to develop a C++ GUI plugin for the
	      ArmarX distributed robot framework. A standout contribution of mine was the
	      design and implementation of a foldable timeline that visually represented the
	      robot's hierarchically structured memory over time,
	      leveraging \href{https://doc.qt.io/qt-5/}{Qt5} for the basic components.
	\item \textbf{Karlsruhe Institute of Technology, Karlsruhe, Germany} \hfill September 2021--December 2021 \\
	      Research assistant at the Chair for Political Economy. Sped up time to run online experiments for the chair from weeks to hours. I set up hosting of the experiments on Heroku and recruitment of participants through Prolific. % Previously the chair had outsorced running the experiments to a different chair.
\end{itemize}

\ifdefined\EAVersion
	\section{Extracurriculars}
	\denseouterlist{
		\entrymid[\textbullet]
		{\textbf{Coorganizer of local EA group Karlsruhe}}{April 2021--November 2023}
		{Booking rooms, went through the paperwork to become an official university group.}
		\entrymid[\textbullet]
		{\textbf{Member in \href{https://aisafety.com/}{AI-Safety-Reading-Group}}}{October 2019--June 2020}
		{Learned about risks from advanced artificial intelligence. Read Human Compatible, Superintelligence and Rationality from AI to Zombies.}
		\entrymid[\textbullet]
		{\textbf{Commenting and occasionally posting on \href{https://www.metaculus.com/accounts/profile/106992/}{Metaculus} and later \href{https://www.lesswrong.com/users/morpheus}{Lesswrong}}}{June 2019--Current} %2019 was starting metaculus
		{\textasciitilde{} 2000 predictions on \textasciitilde{} 600 questions on Metaculus. Currently active only on Lesswrong.}
	}
\fi

\section{Projects}

% \begin{minipage}[t]{0.505\textwidth}

\begin{itemize}
	\item \href{https://sonofhypnos.github.io/blog/prediction/python/2021/01/30/sp500.html}{Predicting the S\&P 500} (Python, Numpy, Pandas), by sampling from past market data. I used this to win money on the \href{https://www.almanisprivate.com/}{Almanis} prediction market. This method worked better for me than just eyeballing trends.
	\item \href{https://github.com/sonofhypnos/fatebook.el}{fatebook.el}. An Emacs plugin to create and upload predictions to Fatebook. I created the tool for my own use.
	      % \item \href{https://github.com/sonofhypnos/fatebook.el}{fatebook.el}. An Emacs plugin to create and upload predictions to Fatebook (I have no relations to Fatebook I just use their product)
\end{itemize}
% \end{minipage}

\end{document}
