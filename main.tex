\documentclass[letterpaper,9pt]{article}

\usepackage[margin=1in]{geometry}
\usepackage{xcolor}
% TODO: put your hackathon on your resume even if you did not win?

% Define colors for dark mode
\definecolor{background}{RGB}{35,35,35}
\definecolor{text}{RGB}{220,220,220}
\definecolor{accent}{RGB}{100,149,237} % Light blue color


\usepackage[colorlinks = true,
        urlcolor  = yellow]{hyperref}

% Set page color and text color
\pagecolor{background}
\color{text}

\usepackage{titlesec}
\titleformat{\section}{\Large\bfseries\color{accent}}{\thesection}{1em}{}[\titlerule]

\begin{document}

% Header
\begin{center}
\Huge\textbf{Tassilo Neubauer}

\vspace{0.5em}
\large
\textcolor{accent}{Email:} \href{tassilo.neubauer@gmail.com}{tassilo.neubauer@gmail.com} \hspace{2em}
\textcolor{accent}{GitHub:} \href{https://github.com/sonofhypnos}{github.com/sonofhypnos}
\end{center}

\section*{Education}
\textbf{Karlsruhe Institute of Technology} \hfill Karlsruhe, Germany \\
B.S. Computer Science, Grade Average 3.1 on a scale from 5 (lowest) to 1 (highest) \hfill October 2020--December 2023
\begin{itemize}
    \item \textit{Thesis:} Machine Learning-based Root Cause Analysis for Intelligent Production.
\end{itemize}

\section*{Courses}
\begin{itemize}
    \item \textbf{Math:} Linear Algebra I \& II, Real Analysis I \& II, Probability Theory, Numerical Math for Computer Scientists
    \item \textbf{Computer Science:} Programming (Java), Software Engineering (Java), Programming Paradigms, Databases, Cognitive Systems, Theory of Computer Science I \& II, Algorithms I \& II, Advanced Algorithmic Programming, Operating Systems, Computer Networks among others
    \item \textbf{Physics:} Experimental Physics I \& II, Modern Physics for Computer Scientists
\end{itemize}

\section*{Experience}
\begin{itemize}
    \item \textbf{Freelance Software Developer} \hfill Online, June 2024--August 2024 \\
        Worked as fullstack developer for an early-stage startup, expanding their MVP:
        \begin{itemize}
            \item Collaborated with senior developer to extend existing codebase:
			\begin{itemize}
				\item Developed chat frontend and payment dashboards
				\item Implemented payment backend functionality
				\item Enhanced chatbot API and integrated RAG for improved search capabilities
			\end{itemize}
        \end{itemize}
    \item \textbf{Freelance Software Developer} \hfill Online, January 2024--June 2024 \\
        Worked on Data pipeline development with Exosome Consulting.
    \item \textbf{Forschungszentrum Informatik, Karlsruhe, Germany} \hfill March 2022--December 2022 \\
          Research assistant focusing on generating visualization recommendations using an
          evolutionary algorithm combined with random forests and simple feedforward neural nets,
          based on artificial human feedback.
    \item \textbf{Karlsruhe Institute of Technology, Karlsruhe, Germany} \hfill September 2021--December 2021 \\
          Research assistant at the Chair for Political Economy. Sped up time to run online experiments for the chair from
          weeks to hours. I set up hosting of the experiments on Heroku and recruitment of participants through Prolific.
\end{itemize}

\section*{Projects}
\begin{itemize}
    \item \textbf{ArmarX Distributed Robot Framework Plugin} \hfill April 2022--September 2022 \\
          Collaborated with a team of four students to develop a C++ GUI plugin for the
          ArmarX distributed robot framework. A standout contribution of mine was the
          design and implementation of a foldable timeline that visually represented the
          robot's hierarchically structured memory over time, leveraging Qt5 for the basic components.
	\item \href{https://sonofhypnos.github.io/blog/prediction/python/2021/01/30/sp500.html}{Predicting the S\&P 500} (Python, Numpy, Pandas), by sampling from past market data. I used this to win money on the \href{https://www.almanisprivate.com/}{Almanis} prediction market. This method worked better for me than just eyeballing trends.
	\item \href{https://github.com/sonofhypnos/fatebook.el}{fatebook.el}. An Emacs plugin to create and upload predictions to Fatebook. I created the tool for my own use.
	      % \item \href{https://github.com/sonofhypnos/fatebook.el}{fatebook.el}. An Emacs plugin to create and upload predictions to Fatebook (I have no relations to Fatebook I just use their product)
\end{itemize}

\end{document}
